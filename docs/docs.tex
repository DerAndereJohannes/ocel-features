\documentclass{article}
\usepackage[utf8]{inputenc}
\usepackage{algorithm2e}
\usepackage{amsmath}

\title{OCEL Feature Extraction Documentation}
\date{November 2021 - ??? 2022}

\begin{document}

\maketitle

\tableofcontents
\pagebreak

\section{Document Format}
The structure section will go over all the structures that are created in order for many of the feature extraction methods to function. Each structure will have information on input, context behind the structure and a sample image to demonstrate the output on a basic example.

The Feature sections go over feature extraction methods explored throughout the thesis on the topic. All features are aimed at being very general, applicable to any input ocel log. The descriptions will include the input, structures used, output, context and a very small basic visualization based on the structure images that are found in the structures section. 

\section{Structures}
\subsection{Object-Based}
\subsection{Event-Based}
\section{Object Based Features}
\subsection{Point wise Features}
\subsection{Local Features}
\subsection{Global Features}
\section{Event Based Features}
\subsection{Point wise Features}
\subsection{Local Features}
\subsection{Global Features}

\end{document}